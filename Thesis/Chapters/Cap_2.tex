\chapter{Introduzione}\label{cap:Programmazione del supporto hardware del progetto}

\section{Programmazione RaspberryPi}\label{sez:Programmazione RaspberryPi}

Come descritto nella sezione precedente, la "board" usata per il progetto è stata un "RaspberryPi 3b+", la quale chiaramente non è una "board" ormai datata, siccome era già da circa un anno che era stato rilasciato il "RaspebrryPi 4". Ragionevolmente il software disponibile per questi sistemi è stato aggiornato per adattarsi alle nuove caratteristiche e potenzialità, con la conseguenza che i modelli più vecchi non sono più idonei a supportare il software "main line" a causa delle risorse limitate rispetto ai più recenti.\\
Per questo motivo si era pensato (pur utilizzando software "main line") di cercare di risparmiare risorse provando a sostituire il "desktop eviroment" di base, installato con il sistema operativo "Raspbian". In particolare l'intento era quello di confrontare le prestazioni tra il "desktop enviroment Pixel" (basato su "LXDE") e di "XFCE".  

\subsection{installazione del "desktop eviroment XFCE"}
Dal sito di "\href{https://www.raspberrypi.com/software/operating-systems/}{RaspberryPi}" è stata scaricata l'immagine del sistema operativo "Raspbian" in formato "Lite" da poter poi "flashare" sulla "micro sd" della "board".  Normalmente non c'è bisogno di scaricare una "distribuzione lite" (quindi senza un "desktop enviroment" installato), in quanto la struttura gerarchia di "linux" tiene i vari servizi isolati, per questo motivo nello stesso sistema operativo si possono avere più "desktop enviroment" installati contemporaneamente, con la possibilità in fase di "Log" di poter scegliere quale caricare. Ma il fatto che i due "desktop enviromet" ("pixel e XFCE") condividessero la stessa radice e che peraltro fossero entrambi basati su un motore grafico ("Xorg 11") che non è famoso per la sua affidabilità, ha portato alla scelta di cui sopra. 
Dopo il download della immagine del sistema operativo scelto, quest'ultimo è stato "flashato" sulla "micro sd" utilizzando l'applicazione "\href{https://etcher.balena.io/}{Balenaetcher}", mentre per fare i backup del sistema e il ripristino è stato usato "\href{https://www.tweaking4all.com/software/macosx-software/applepi-baker-v2/}{ApplePi-Baker v2}". \\
Una volta aver avviato e configurato il sistema operativo dopo il primo avvio sul "RaspberryPi", si è poi proceduto alla installazione del "desktop enviroment XFCE" utilizzando questa \href{https://www.makeuseof.com/desktop-environments-you-can-run-on-a-raspberry-pi/}{guida}.\\
