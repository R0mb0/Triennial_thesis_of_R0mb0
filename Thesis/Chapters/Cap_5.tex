Supponendo che non vi siano problemi e che l'autenticazione dell'utente avvenga senza difficoltà, ora si può iniziare a discutere delle pagine che amministrano il vero e proprio servizio della "web app".\\
Nella pagina di accesso alle varie esercitazioni disponibili, non c'è molto cose da discutere se non il fatto che questa pagina amministra la sessione, controllando e impostando le varie variabili poi amministrare dalle pagine delle varie esercitazioni. 

\paragraph{Estratto di Entry.php}\leavevmode\\

\begin{lstlisting}[language=php]
	/*format all session variable*/
	$_SESSION['permission']        = null;
	$_SESSION['numbers']           = null;
	$_SESSION['topic']             = null;
	$_SESSION['questions']         = null;
	$_SESSION['maxQuestions']      = null;
	$_SESSION['score']             = null;
	$_SESSION['answeredQuestions'] = null;
	
	/*control if user is logged*/
	if($_SESSION['mail'] !== null && $_SESSION['password'] !== null){
		
		$query = $utilities->getMysql()->query("SELECT password FROM user_table1 WHERE (email = '{$_SESSION['mail']}')");
		$tempArray = $query->fetch_array(MYSQLI_ASSOC);
		$password = $tempArray['password'];
		
		if($_SESSION['password'] !== $password){
			$_SESSION['user']     = "NotAllow";
			$_SESSION['mail']     = null;
			$_SESSION['password'] = null;
			header("Location: ../main.php");
			exit;
		}
		
	}else{
		$_SESSION['user']     = "NotAllow";
		$_SESSION['mail']     = null;
		$_SESSION['password'] = null;
		header("Location: ../main.php");
		exit;
	}
\end{lstlisting}
Dal codice incluso si può inoltre notare che viene controllato, osservando le variabili di sessione, che l'accesso ai vari corsi sia reso disponibile ai soli utenti registrati. A questo punto può risultare contraddittorio rispetto al caso prima espresso, che in questa sede si tengano in sessione le credenziali dell'utente, ma si vuole far notare che le credenziali tenute in sessione non sono in chiaro (caso richiesto nella condizione espressa precedentemente) e che inoltre lo "spoofing" a questo livello non presenterebbe un problema, in quanto il caso peggiore che potrebbe accadere è che qualcuno possa accedere alle esercitazioni con le  credenziali di un'altra persona senza la possibilità di interagire con l'utente al quale sono state "rubate" le credenziali di autenticazione.\\

Ora in ordine di rilevanza verranno presentati i file di ogni esercitazione, mettendo in evidenza le peculiarità di ognuno. I primi file saranno riportati maggiormente nella loro interezza in modo da esplicitare il metodo con il quale è stata condotta la costruzione dei restanti. Per poi arrivare a riportare di ogni file solo le parti salienti.\\

\paragraph{Simulazione elementi di carteggio}\leavevmode\\

\begin{lstlisting}[language=php]
	if($_SESSION['permission'] !== "true"){
		/*control if user is logged*/
		if($_SESSION['mail'] !== null && $_SESSION['password'] !== null){
			
			$query = $utilities->getMysql()->query("SELECT password FROM user_table1 WHERE (email = '{$_SESSION['mail']}')");
			$tempArray = $query->fetch_array(MYSQLI_ASSOC);
			$password = $tempArray['password'];
			
			if($_SESSION['password'] !== $password){
				$_SESSION['user']     = "NotAllow";
				$_SESSION['mail']     = null;
				$_SESSION['password'] = null;
				header("Location: ../../main.php");
				exit;
			}
			/*in case of page reload*/
			$_SESSION['permission'] = "true";
			
		}else{
			$_SESSION['user']     = "NotAllow";
			$_SESSION['mail']     = null;
			$_SESSION['password'] = null;
			header("Location: ../../main.php");
			exit;
		}
	}
	
	/*find course properties*/
	$query = $utilities->getMysql()->query("SELECT * FROM charting_elements_properties WHERE (id = '1')");
	$tempArray = $query->fetch_array(MYSQLI_ASSOC);
	$maxQuestions = $tempArray['questions'];//<--- Very important field
	$maxErrors = $tempArray['errors'];//<--- Very important field
	
	/*find questions number*/
	$query = $utilities->getMysql()->query("SELECT COUNT(*) FROM charting_elements");
	$tempArray = $query->fetch_array(MYSQLI_ASSOC);
	$questionsNumber = $tempArray['COUNT(*)'];//<--- Very important field
	
	
	
	/*--------- GENERATE INDEXES AND EXTRACT QUESTIONS ----------*/
	
	$indexNumbers = array();
	$questions = array();
	
	$temp = 0;
	
	while($temp < $maxQuestions){
		
		if($_SESSION['numbers'][0] !== null){
			// if i'm here meaning that the page has been reloaded     
			if($indexNumbers[0] !== null){
				//generate number in recursive case
				$tempNum = random_int(1, $questionsNumber);
				//verify that the number isn't duplicated
				while(in_array($tempNum,$indexNumbers) || in_array($tempNum,$_SESSION['numbers'])){
					$tempNum = random_int(1, $questionsNumber);
				}
				
				//insert number
				$indexNumbers[$temp] = $tempNum; 
				
			}else{
				//generate number on base case
				$tempNum = random_int(1, $questionsNumber);
				//verify that the number is new (respect the past)
				while(in_array($tempNum,$_SESSION['numbers'])){
					$tempNum = random_int(1, $questionsNumber);
				}
				//insert number
				$indexNumbers[$temp] = $tempNum; 
			}
			
		}else{
			//if i'm here meaning that the page is new
			if($indexNumbers[0] !== null){
				//generate number in recursive case
				$tempNum = random_int(1, $questionsNumber);
				//verify that the number isn't duplicated
				while(in_array($tempNum,$indexNumbers)){
					$tempNum = random_int(1, $questionsNumber);
				}
				
				//insert number
				$indexNumbers[$temp] = $tempNum;
				
			}else{
				//generate number on base case
				$indexNumbers[$temp] = random_int(1, $questionsNumber);
			}
		}
		
		/*extract questions from db using the index generated*/
		$query = $utilities->getMysql()->query("SELECT * FROM charting_elements WHERE (id = '{$indexNumbers[$temp]}')");
		$tempArray = $query->fetch_array(MYSQLI_ASSOC);
		
		$questions[$temp] = array(
		"question_text"   => $tempArray['question_text'],
		"question1"       => $tempArray['question_1'],
		"question2"       => $tempArray['question_2'],
		"question3"       => $tempArray['question_3'],
		"question4"       => $tempArray['question_4'],
		"question5"       => $tempArray['question_5'],
		"answer1"         => $tempArray['answer_1'],
		"answer2"         => $tempArray['answer_2'],
		"answer3"         => $tempArray['answer_3'],
		"answer4"         => $tempArray['answer_4'],
		"answer5"         => $tempArray['answer_5'],
		);
		
		++$temp;
	}
	
	$_SESSION['numbers'] = $indexNumbers;
\end{lstlisting}

Come si può facilmente vedere, una delle prime che vengono fatte in assoluto è sempre il controllo dell'utente, per evitare l'accesso non autorizzato al servizio. Da notare inoltre che grazie alla variabile di sessione "permission" si garantisce che l'utente non venga espulso nel caso in cui ricarichi la pagina.\\
Successivamente si procede con l'estrazione dal database delle proprietà della simulazione, come il numero di domande da somministrare e il numero di errori massimi da poter commettere. Viene poi fatto il calcolo del numero delle domande presenti nel database in modo da poter stabilire il "range" per la selezione causale delle domande da presentare. Il fatto di contare le domande presenti nel database garantisce che ogni volta che viene inserita una nuova domanda nel database, questa sia subito disponibile senza dover aggiornare dei parametri che in questo caso sono deducibili. In questo modo viene anche semplificato il mantenimento e aggiornamento delle domande.\\
A questo punto è possibile estrarre casualmente i vari indici delle domande da presentare. Questa operazione viene fatta con un ciclo "while" annidato che "cicla" fino a che non ha generato senza ripetizioni (anche nei confronti dell'ultima esercitazione fatta) tutti gli indici richiesti. Per cercare di tenere alto il livello di efficienza delle operazioni l'ultima parte del ciclo "while" è stata preposta alla estrazione ed inserimento in un "array" delle domande, in modo da poter risparmiare il fatto di dover fare un secondo ciclo.\\

Il codice "html" riportato di seguito serve per mostrare come le informazioni estratte nella parte di "php" sono state integrate nella parte visibile della pagina. In generale si è sempre usato il sistema di interrompere il codice "html" richiamando il "tag" del "php" e di usare il comando "echo" per riportare le informazioni necessarie.\\

\begin{lstlisting}[language=html]
	<!DOCTYPE html>
	<html lang="it" >
	<head>
	<meta charset="UTF-8">
	<title>Simulazione su elementi di carteggio nautico</title>
	<link rel='stylesheet' href='https://fonts.googleapis.com/css?family=Rubik:400,700'><link rel="stylesheet" href="simulationStyle.css">
	
	</head>
	<body>  
	
	<div class="welcomeText">
	<h1>Quiz patenti nautiche DD 131 del 31 Maggio 2022</h1>
	<h2>Simulazione sulle domande su elementi di carteggio nautico</h2>
	</div>
	
	<div class="question-form">
	<form method="post">
	<h1><?php
	if($indexNumbers[1] !== null){
		echo "Indice domande: ";
		$temp = 0;
		
		while($temp < $maxQuestions){
			echo $indexNumbers[$temp]." ";
			++$temp;
		}
		
	}else{
		echo "Indice domanda: ".$indexNumbers[0];
	}
	?>
	</h1>
	<div class="content">
	<?php 
	$temp = 0;
	while($temp < $maxQuestions){
		echo "<b>".$questions[$temp]["question_text"]."</b><br>";
		echo"<br>";
		echo $questions[$temp]["question1"]."<br>";
		echo $questions[$temp]["question2"]."<br>";
		echo $questions[$temp]["question3"]."<br>";
		echo $questions[$temp]["question4"]."<br>";
		echo $questions[$temp]["question5"]."<br>";
		echo"<br>";
		++$temp;
	}
	?> 
	</div>
	<div class="actions">
	<input type="button" name="revision" value="Guarda la soluzione" id="revision" onclick="check()"/>
	<div id="chart">Non hai la carta nautica? <a href="../Nautical_Charts/Carta_Nautica_5D.pdf" download> Scaricala qui</a></div>
	</div>
	<div id="answer">
	<?php
	$temp = 0;
	while($temp < $maxQuestions){
		if($indexNumbers[1] !== null){
			$tempNumber = $temp + 1;
			echo"<h3>Soluzione domanda: ".$tempNumber." </h3><br>"; 
		}
		echo"Soluzione quesito 1: <b>".$questions[$temp]["answer1"]."</b><br>";
		echo"Soluzione quesito 2: <b>".$questions[$temp]["answer2"]."</b><br>";
		echo"Soluzione quesito 3: <b>".$questions[$temp]["answer3"]."</b><br>";
		echo"Soluzione quesito 4: <b>".$questions[$temp]["answer4"]."</b><br>";
		echo"Soluzione quesito 5: <b>".$questions[$temp]["answer5"]."</b><br>";
		echo"<br>";
		++$temp;
	}
	?>
	<h3><?php
	if($maxErrors > 1){
		echo "La prova è stata superata se si sono commessi massimo ".$maxErrors." errori.";
	}else{
		echo "La prova è stata superata se è stato commesso massimo ".$maxErrors." errore.";  
	}
	
	?></h3>
	</div>
	<div class="foot">
	<input type="submit" name="cancel" value="Esci dalla simulazione" id="button2"/>
	<input type="button" name="nextPage" value="Prova successiva" id="button1" onclick="next()"/>
	</div>
	</form>
	</div>
\end{lstlisting}

Un problema che però si evidenzia a questo punto è che serve un sistema per poter nascondere la risposta al quesito, in modo da permettere l'esercizio. Questa cosa non la si può far se non con il "javascript" che permette lato "front-end" di animare le varie parti del "html".\\
Qui di seguito è riportato lo "script" che è stato scritto per permettere questa funzionalità.\\

\begin{lstlisting}[language=javascript]
	<script>
	
	/*hide elements at the beginning*/
	var x = document.getElementById("answer");
	x.style.display = "none";
	
	/*show elements when the button is clicked*/
	function check(){
		var x = document.getElementById("answer");
		
		if(x.style.display === "none"){
			x.style.display = "block";
		}else{
			x.style.display = "none";
		}	
	}
	
	function next(){
		location.reload();  
	}
	</script>
\end{lstlisting}

La funzione "check" dichiarata nella parte di "scripting" viene evocata dal pulsante identificato come "revision", e permette di mostrare o di nascondere la parte in "html" contrassegnata come "anser". La quale inizialmente viene resa nascosta.\\
Siccome ogni esercitazione viene generata nella parte di codice in "php" e che questa viene eseguita soltanto durante il caricamento della pagina, come ultima cosa è stata dichiarata una funzione che se invocata dal pulsante "nextPage" forza il "reloading" della pagina, generando di fatto (nelle modalità vista prima) una nuova esercitazione.\\

Ora che è stata mostrata la simulazione, si vuole mostrare l'esercitazione riferita sempre allo stesso tipo di domande. Incominciando come di consueto ad analizzare il codice in "php".
\paragraph{Esercitazione elementi di carteggio}

\begin{lstlisting}[language=php]
	
	if($_SESSION['permission'] !== "true"){
		/*control if user is logged*/
		if($_SESSION['mail'] !== null && $_SESSION['password'] !== null){
			
			$query = $utilities->getMysql()->query("SELECT password FROM user_table1 WHERE (email = '{$_SESSION['mail']}')");
			$tempArray = $query->fetch_array(MYSQLI_ASSOC);
			$password = $tempArray['password'];
			
			if($_SESSION['password'] !== $password){
				$_SESSION['user']     = "NotAllow";
				$_SESSION['mail']     = null;
				$_SESSION['password'] = null;
				header("Location: ../../main.php");
				exit;
			}
			/*in case of page reload*/
			$_SESSION['permission'] = "true";
			
		}else{
			$_SESSION['user']     = "NotAllow";
			$_SESSION['mail']     = null;
			$_SESSION['password'] = null;
			header("Location: ../../main.php");
			exit;
		}
	}
	
	/*---------- END USER VERIFY ----------*/
	
	/*set count variable*/
	if($_SESSION['numbers'] == null){
		$_SESSION['numbers'] = 1;
	}
	
	
	/*find questions number*/
	$query = $utilities->getMysql()->query("SELECT COUNT(*) FROM charting_elements");
	$tempArray = $query->fetch_array(MYSQLI_ASSOC);
	$maxQuestions = $tempArray['COUNT(*)'];//<--- Very important field
	
	
	/*---------- START BUTTONS PART ----------*/
	
	if(isset($_POST['prevPage'])){
		if($_SESSION['numbers'] - 1 >= 1 && $_SESSION['numbers'] !== null){
			--$_SESSION['numbers'];
		}else{
			$utilities->Popup("Non è possibile visualizzare la domanda precedente");
		}
	}
	
	if(isset($_POST['cancel'])){
		$_SESSION['permission'] = null;
		$_SESSION['numbers']     = null;
		header("Location: ../entry.php");
		exit;
	}
	
	if(isset($_POST['nextPage'])){
		if($_SESSION['numbers'] + 1 <= $maxQuestions && $_SESSION['numbers'] !== null){
			++$_SESSION['numbers'];
		}else{
			$utilities->Popup("Non è possibile visualizzare la domanda successiva");
		}
	}
	
	/*---------- GENERATE QUESTIONS ----------*/
	
	$query = $utilities->getMysql()->query("SELECT * FROM charting_elements WHERE (id = '{$_SESSION['numbers']}')");
	$tempArray = $query->fetch_array(MYSQLI_ASSOC);
	$question_text = $tempArray['question_text'];
	$question1     = $tempArray['question_1'];
	$question2     = $tempArray['question_2'];
	$question3     = $tempArray['question_3'];
	$question4     = $tempArray['question_4'];
	$question5     = $tempArray['question_5'];
	$answer1       = $tempArray['answer_1'];
	$answer2       = $tempArray['answer_2'];
	$answer3       = $tempArray['answer_3'];
	$answer4       = $tempArray['answer_4'];
	$answer5       = $tempArray['answer_5'];
\end{lstlisting}

Come si può leggere, anche in questo caso la prima cosa che viene fatta è sempre il controllo dell'utente (da questo punto non più riportato nei codici avvenire).  Poi nel caso di una nuova esercitazione viene instanziata una variabile di sessione che tiene il conto dell'indice della domanda corrente, in modo che durante il "reload" si possa accedere alla domanda successiva o alla precedente se è possibile. Si può notare che la variabile viene sempre instanziata con il valore di 1, questo perché per specifica le domande sono conteggiate a partire dal numero 1 e non dallo 0 come abitudine informatica.\\
Saltata la parte che riguarda la gestione dei pulsanti perché di poco interesse, si mostra come in questo caso, non dovendo gestire più di una domanda si è preferito dichiarare tante variabili quante le informazioni necessarie, al posto di dichiarare un "array" come nel caso precedente.\\
Lampante a questo punto sarà la semplicità con la quale le informazioni sono state riportate nella parte visibile della pagina. Ma siccome ancora non è stato mostrato il codice "html" tipico delle esercitazioni, anche in questo caso viene riportato nella sua interezza, per dar modo di comprendere questo aspetto.\\

\begin{lstlisting}[language=php]
\end{lstlisting}