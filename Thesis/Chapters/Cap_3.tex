\chapter{Programmazione del servizio online}\label{cap:Programmazione del servizio online}

\section{Configurazione database}\label{sez:Configurazione database}

Seguendo la documentazione sul sito di "MySQL.com" inizialmente è stato creato un account con il quale amministrare il database del progetto. Questa fase è stata un pò difficoltosa in quanto l'applicazione "phpMyAdmin" ogni tanto dava  problemi a riconoscere il lavoro fatto da riga di comando (come ad esempio l'assegnazione dei permessi), ma dopo alcuni tentativi, la configurazione è andata a buon fine.\\
Successivamente mi sono impegnato nella costruzione di un database efficiente seguendo le linee guida scritte nella \hyperref[cap:Introduzione]{introduzione}.\\
In questo caso specifico non sono state fatte ulteriori operazioni, come per esempio la definizione dei vincoli d'integrità, in quanto, i dati da contenere all'interno del database non necessitano di veri e propri vincoli d'integrità. Ad esempio l'unico vicolo d'integrità 