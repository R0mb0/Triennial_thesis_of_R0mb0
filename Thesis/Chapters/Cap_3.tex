\chapter{Programmazione del servizio online}\label{cap:Programmazione del servizio online}

\section{Configurazione database}\label{sez:Configurazione database}

Seguendo la documentazione sul sito di "MySQL.com" inizialmente è stato creato un account con il quale amministrare il database del progetto. Questa fase è stata un pò difficoltosa in quanto l'applicazione "phpMyAdmin" ogni tanto dava  problemi a riconoscere il lavoro fatto da riga di comando (come ad esempio l'assegnazione dei permessi), ma dopo alcuni tentativi, la configurazione è andata a buon fine.\\
Successivamente mi sono impegnato nella costruzione di un database efficiente seguendo le linee guida scritte nella \hyperref[cap:Introduzione]{introduzione}.\\
In questo caso specifico non sono state fatte ulteriori operazioni, come per esempio la definizione dei vincoli d'integrità, in quanto, i dati da contenere all'interno del database non necessitano di veri e propri vincoli d'integrità.\\

\subsection{Instanzazione delle tabelle dei quiz}
Le tabelle dei quiz sono state istanziate seguendo un ordine di complessità, ovvero, non si è seguito l'ordine cronologico con il quale le domande sono state rilasciate, ma si è partito dai quiz più semplici da tradurre per poi passare a quelli più complessi.\\
\begin{itemize}
	\item \textbf{Elementi di carteggio}\\
	Seguendo le linee guide ragionate in fase di  \hyperref[cap:Introduzione]{pianificazione}, sono state istanziate due tabelle: "charting_elements" e  "charting_elements_properties". La prima serve per contenere le varie domande e risposte, mentre la seconda ha il compito di definire le proprietà dell'esame. Inizialmente si pensava di poter scorporare le domande dal testo della domanda con l'intuizione che le domande fossero uguali per tutti i testi (introducendo quindi una terza tabella), ma poi dopo una approfondita analisi ci si è resi conto che alcune domande hanno delle piccole variazioni 
	Più nello specifico, la prima tabella presenta le seguenti colonne: "id" (il numero di domanda), "question_text", "question_1", "question_2", "question_3", "question_4", "question_5","answer_1", "answer_2", "answer_3", "answer_4" e "answer_5". Si segnala inoltre che in questo caso le domande sono state rimaneggiate durante l'inserimento nel database per aumentare la chiarezza di esposizione del testo, in particolare il testo vero e proprio della domanda è stato messo in grassetto aggiungendo i "tag del html" direttamente nel testo, mentre il settore di riferimento è stato lasciato in maiuscolo così come riportato nella documentazione. La seconda tabella invece è composta da queste colonne: "id", "questions", "errors", che descrivono rispettivamente: l'identificativo di riga, tenuto per poter in caso fare dei cambiamenti provvisori alle proprietà senza dover modificare le originali; il numero di domande da presentare durante la simulazione; il numero massimo si errori concessi per il superamento della prova. 
	
	\item \textbf{Carteggio}\\
	Per modellare questo genere di domande sono state formate tre tabelle: "charting_test_5d", "charting_test_42d" e "charting_test_properties". La prima serve per contenere  le domande di carteggio riguardanti la carta nautica "5d", la seconda, quelle della carta "42d" e la terza contiene le proprietà dell'esame.\\
	Più nello specifico la prima tabella si compone delle seguenti colonne: "id", "question_text", "answer", "area" e "topic", che rispettivamente descrivono: il numero della domanda; il testo della domanda; la risposta della domanda; la zona di riferimento (le carte nautiche si dividono in 4 quadranti, rispettivamente: "nord-ovest", "nord-est", "sud-ovest" e sud-est); l'argomento delle domande.  Mentre la seconda tabella è composta da: "id" (anche in questo casi tenuto come identificatore di riga), "5d_questions", "42d_questions" e "errors", che serve per descrivere il numero di domande da presentare nel caso della carta "5d" e  "42d" oltre che il numero massimo di errori consentiti. \\
	
	\item \textbf{Quiz base}\\
	Anche in questo caso sono state due tabelle: "base_quiz" e "base_quiz_properties", come intuibile, la prima tabella contiene le domande e la seconda le proprietà dell'esame.\\
	Andando più in dettaglio, la prima tabella si compone delle seguenti colonne: "id", "img",  "question_text", "answer_text_1", "answer_1", "answer_text_2", "answer_2", "answer_text_3", "answer_3" e "topic", con in breve queste funzioni; l'"Id" rappresenta il numero della domanda, in "img" viene salvata l'immagine correlata alla domanda, "topic" rappresenta invece l'argomento di riferimento della domanda, mentre il resto sono la domanda con le varie risposte e la veridicità di quest'ultime. Si vuole fare notare al fatto che pare logico pensare che su tre risposte ci sia una sola riposta giusta e che quindi sarebbe stato più ragionevole salvare nel database solo il numero della risposta giusta. Ma il problema è che nella documentazione del ministero non è esplicitamente segnalata questa affermazione, per tanto, si è deciso di salvare i dati così come presentati nella documentazione,  nei confronti di una ottica futura nella quale potrebbe capitare che solo una risposta su tre sia sbagliata.\\
	La seconda tabella, che rappresenta il numero di domande da presentare per ogni argomento e il numero massimo di errori concessi è composta da: "id", "teoria_dello_scafo_questions", "motori_questions", "sicurezza_della_navigazione_questions", "manovra_e_condotta_questions", "colreg_e_segnalamento_marittimo_questions", "meteorologia_questions", "navigazione_cartografica_ed_elettronica_questions", "normativa_diportistica_e_ambientale_questions" e  "errors".\\
	
	\item \textbf{Quiz Vela}
	Come di consueto sono state create due tabelle con la funzione di contenere le domande, le risposte e le proprietà della prova. Le tabelle sono state nominate come: "ship_boat_quiz" e "ship_boat_properties"\\
	La prima è composta da: "id", "img", "question_text", "answer" e "topic", le funzioni dei campi sono gli stessi del caso precedente, ma in particolare in "answer" è stata modellata una condizione booleana, siccome per specifica è stato esplicitato che in questo caso ogni domanda può essere vera o falsa, per questo rispetto alla documentazione è stata eliminata una ridondanza (siccome nella documentazione venivano esplicitati entrambi i casi booleani). Mentre la seconda: "id", "questions" e "errors", anche in questo caso con funzioni identiche rispetto alle precedenti presentate.  
\end{itemize}

Per evitare di ripetersi, si segnala a questo punto che tutti i campi "id" (che fanno da super chiave) sono stati come "int", ogni campo di testo è stato trattato come "text", la condizione booleana è stata trattata come "char" (V o F) e le immagini sono state trattate come "blob". 

\subsection{Instanzazione della tabella utente}
La tabella utente è stata formata con i seguenti campi: "Id" (come super chiave), "name", "surname", "email", "password", "verified" e "admin". Tutti i dati sensibili non sono salvati in chiaro all'interno del database, ma le il testo viene direttamente cifrato e decifrato dalla applicazione online. Il campo "id" è trattato come "int", invece i restanti fino a "verified" sono stati tutti trattati come "text" e servono per salvare le informazioni di cui portano il nome. "verified" è invece un booleano puro (quindi "boolean") usato per tenere in memoria se la mail dell'utente è stata più o meno verificata (di default il valore è 0), l'ultimo campo invece è stato introdotto in previsione di una pagina di amministrazione che non è stata implementata, per questo motivo è un campo ancora inutilizzato. 