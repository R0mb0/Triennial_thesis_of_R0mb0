\chapter{Programmazione del servizio online}\label{cap:Programmazione del servizio online}

\section{Configurazione database}\label{sez:Configurazione database}

Seguendo la documentazione sul sito di "MySQL.com" inizialmente è stato creato un account con il quale amministrare il database del progetto. Questa fase è stata un pò difficoltosa in quanto l'applicazione "phpMyAdmin" ogni tanto dava  problemi a riconoscere il lavoro fatto da riga di comando (come ad esempio l'assegnazione dei permessi), ma dopo alcuni tentativi, la configurazione è andata a buon fine.\\
Successivamente mi sono impegnato nella costruzione di un database efficiente seguendo le linee guida scritte nella \hyperref[cap:Introduzione]{introduzione}.\\
In questo caso specifico non sono state fatte ulteriori operazioni, come per esempio la definizione dei vincoli d'integrità, in quanto, i dati da contenere all'interno del database non necessitano di veri e propri vincoli d'integrità.\\

\subsection{Instanzazione delle tabelle dei quiz}
Le tabelle dei quiz sono state istanziate seguendo un ordine di complessità, ovvero, non si è seguito l'ordine cronologico con il quale le domande sono state rilasciate, ma si è partito dai quiz più semplici da tradurre per poi passare a quelli più complessi.\\
\begin{itemize}
	\item \textbf{Elementi di carteggio}\\
	Seguendo le linee guide ragionate in fase di  \hyperref[cap:Introduzione]{pianificazione}, sono state istanziate 3 tabelle: "charting_elements",  "charting_elements_properties" e "charting_elements_questions". La prima serve per contenere le varie domande e risposte, mentre la seconda ha il compito di definire le proprietà dell'esame e la terza presenta semplicemente le domande da presentare. Le domande sono state "scorporate" dal testo della domanda, siccome per la conformazione del tipo di prova, bisogna rispondere sempre alle stesse domande in riferimento ai dati presenti nel testo della domanda. Questa affermazione è esplicitata nei testi di documentazione dell'esame.\\ 
	Più nello specifico, la prima tabella presenta le seguenti colonne: "id" (il numero di domanda), "question_text", "answer_1", "answer_2", "answer_3", "answer_4" e "answer_5". Si segnala inoltre che in questo caso le domande sono state rimaneggiate durante l'inserimento nel database per aumentare la chiarezza di esposizione del testo, in particolare il testo vero e proprio della domanda è stato messo in grassetto aggiungendo i "tag del html" direttamente nel testo, mentre il settore di riferimento è stato lasciato in maiuscolo così come riportato nella documentazione. La seconda tabella invece è composta da queste colonne: "id", "questions", "errors", che descrivono rispettivamente: l'identificativo di riga, tenuto per poter in caso fare dei cambiamenti provvisori alle proprietà senza dover modificare le originali; il numero di domande da presentare durante la simulazione; il numero massimo si errori concessi per il superamento della prova. L'ultima tabella presenta le colonne: "id" (per la stessa motivazione della tabella precedente), "question_1", "question_2", "question_3", "question_4" e "question_5".  
	
	\item \textbf{Carteggio}\\
	Anche per modellare questo genere di domande sono state formate 3 tabelle: "charting_test_5d", "charting_test_42d" e "charting_test_properties". La prima serve per contenere  le domande di carteggio riguardanti la carta nautica "5d", la seconda, quelle della carta "42d" e la terza contiene le proprietà dell'esame.\\
	Più nello specifico la prima tabella si compone delle seguenti colonne: "id", "question_text", "answer", "area" e "topic", che rispettivamente descrivono: il numero della domanda; il testo della domanda; la risposta della domanda; la zona di riferimento (le carte nautiche si dividono in 4 quadranti, rispettivamente: "nord-ovest", "nord-est", "sud-ovest" e sud-est), riportata a livello di database come dato, ma questo non viene poi presentato durante la prova, siccome è verosimile che gli esaminatori pretendano che il candidato abbia dimestichezza con le carte nautiche 
\end{itemize}