\chapter{Programmazione del servizio online parte 2}\label{cap:Programmazione del servizio online parte 2}

\section{Creazione del sito internet}\label{sez:Creazione sito internet}

La creazione del dito internet si può suddividere nella creazione delle singole pagine che lo compongo.

\subsection{Creazione pagina di registrazione e di accesso}
L'intera produzione del sito internet è stata fatta seguendo la filosofia di dare più importanza ai lavori "più grossi", facendosi che quindi il progetto si sia sviluppo a partire da un nucleo principale per poi espandersi, al contrario di  come comunemente vengono condotti questo genere di progetti, partendo dalle specifiche più esterne (in modo da avere più velocemente una idea del prodotto finito) per poi convergere verso il nucleo.\\
Secondo la specifica del progetto, il sito deve essere in grado di essere accessibile solo agli utenti registrati (verificando inoltre che la mail che abbiano inserito sia reale), garantendo inoltre la sicurezza, ossia che il sito deve possedere le principali difese per non essere attaccato e che in caso di attacco i dati degli utenti siano all'interno del "database" cifrati in modo da non poter essere utilizzati da terzi.\\
La prima pagina che è stata creata in assoluto è quella di accesso, dalla quale poi premendo un  pulsante è possibile accedere alla pagina di registrazione. Com'è deducibile queste due pagine sono state in contemporanea siccome la prima permette di verificare li corretto funzionamento della seconda.\\

Dopo aver studiato velocemente "l'html" e il "php" (il "javascript" è stato lasciato in secondo piano) decido di sviluppare il codice ("php") con paradigma ad oggetti, siccome mi è molto famigliare. Ragionando solo a livello di logica decido di creare una classe "utilities" nella quale poter scrivere i metodi che poi mi torneranno utili durante anche lo sviluppo delle restanti parti.\\

Si esplicita in questa sede che le varie classi che saranno presentate sono state divise ed ordinate in cartelle sulla base delle loro funzionalità.\\

\paragraph{Classe utilities}\leavevmode\\
La classe "utilities" (quindi il relativo file) è stata implementata in questo modo:\\
Per prima si è deciso di dividere in file separati i metodi e le chiavi per il loro funzionamento. Questa cosa è stata fatta prevalentemente per ragioni di ordine del codice e per aumentare la sicurezza.\\
Il primo metodo presente è quello che permette di creare l'oggetto "mysqli" che serve per poter interagire con il database (da notare come nella porzione di codice inferiore non vi siano riportati direttamente i dati di accesso). Essendo evidente che questo oggetto è per sua stessa statico, durante l'implementazione del metodo si è usato il "singleton pattern" per evitare che in memoria vi siano multipli oggetti identici.\\

\begin{lstlisting}[language=php]
	/*using singleton pattern*/
	public function getMysql(){
		if($this->mysqli == null){
			$this->mysqli = new mysqli(SERVER, USER, PASSWORD, DATABASE);
			
		}
		return $this->mysqli;
	}
\end{lstlisting}

In questa classe sono stati scritti i metodi per cifrare e decifrare le stringe da salvare poi nella tabella degli utenti all'interno del database.\\

\begin{lstlisting}[language=php]
/*Function to crypt a string*/
public function Encrypt($string){
	$encrypt_method = "AES-256-CBC";
	$key = hash( 'sha256', $secret_key );
	$iv = substr( hash( 'sha256', $secret_iv ), 0, 16 );
	$output = base64_encode( openssl_encrypt( $string, $encrypt_method, $key, 0, $iv ) );
	return $output;
}


/*Function to decrypt a string*/
public function Decrypt($string){
	$encrypt_method = "AES-256-CBC";
	$key = hash( 'sha256', $secret_key );
	$iv = substr( hash( 'sha256', $secret_iv ), 0, 16 );
	$output = openssl_decrypt( base64_decode( $string ), $encrypt_method, $key, 0, $iv );
	return $output;
}
\end{lstlisting}

Sempre in questa classe è stato scritto il metodo che si occupa di inviare agli utenti la mail con il link per poter effettuare la verifica della mail di registrazione.\\
Per la costruzione del metodo sono state usate le classi messe a disposizione da "\href{https://github.com/PHPMailer/PHPMailer}{PHPMailer}"

\begin{lstlisting}[language=php]
public function SendEmail($address, $subject, $body, $altBody){
	
	$mail = new PHPMailer(true);
	
	try{
		//server settings
		$mail->SMTPDebug = 0; //SMPT::DEBUG_OFF;
		$mail->isSMTP();
		$mail->Host       = 'smtp.gmail.com';
		$mail->SMTPAuth   = true;
		$mail->Username   = 'quiz.patenti.nautiche@gmail.com';
		$mail->Password   = MAIL_PASSWORD;
		$mail->SMTPSecure = PHPMailer::ENCRYPTION_SMTPS;
		$mail->Port       = 465;
		
		//Recipients
		$mail->setFrom('quiz.patenti.nautiche@gmail.com', 'Quiz Patenti Nautiche');
		$mail->addAddress($address);
		
		//Content
		$mail->isHTML(true);
		$mail->Subject = $subject;
		$mail->Body = $body;
		$mail->AltBody = $altBody;
		
		//Send
		$mail->send();
		
		echo "Message has been sent";
	}catch (Exception $e){
		echo "Message could not be sent. Mailer Error: {$mail->ErrorInfo}";
	}
}
\end{lstlisting}

\textbf{Approfondimento gestione della verifica della mail di registrazione}\\
A questo punto si desidera aprire una breve parentesi per approfondire la gestione della verifica della mail.\\
Dalla classe "Utilities" è possibile usare il metodo che permette l'invio della mail di verifica, ma ovviamente questo metodo prevede degli argomenti in ingresso che adesso saranno esplicitati in ordine: Il primo è l'indirizzo mail del destinatario, che è prelevabile dal database; l'oggetto della mail e i restanti argomenti sono stati definiti all'interno di un'altra classe chiamata "MailUtilities".\\

\paragraph{classe MailUtilities}\leavevmode\\

Come prima cosa è stato definito un "link base di rientro" che se cliccato dalla mail permette di ritornare alla pagina della applicazione web.\\

 \begin{lstlisting}[language=php]
	const URL = "https://quizpatentinautiche.dynamic-dns.net:8080/";
 \end{lstlisting}
 
  Successivamente è stato stabilito un metodo con il quale poter reperire l'oggetto della mail, del quale si è preferito scriverlo in questa sede per tenere raggruppati i vari argomenti necessari per l'invio della mail in un unico posto.\\
 
\begin{lstlisting}[language=php]
public function getSubject(){
	return "Verifica della mail.";
}
\end{lstlisting}

Di seguito è stato definito il corpo "html" della mail, il quale per un qualche motivo poco chiaro non viene visualizzato correttamente su alcuni gestori di servizi mail. \\

\begin{lstlisting}[language=php]
public function getBody($linkToUse){
	return '<html>
	<head>
	<meta http-equiv="Content-Type" content="text/html; charset=utf-8" />
	</head>
	<body style="background-color: #e9ecef;">
	<div class="login-form" style="background: #fff;
	width: 500px; margin: 65px auto; display: -webkit-box;
	display: flex; flex-direction: column; border-radius: 4px;
	box-shadow: 0 2px 25px rgba(0, 0, 0, 0.2);">
	<form method="post">
	<h1 style="padding: 35px 35px 0 35px; font-weight: 300;
	font-family: "Rubik", sans-serif; text-align: center;">
	Verifica la tua mail
	</h1>
	<div style="margin: 25px auto; padding: 18px;
	font-family: "Rubik", sans-serif; text-align: center;">
	Clicca su questo pulsante per accedere alla pagina di verifica della mail inserita su:
	<a href='.self::URL.'>Quiz Patenti Nautiche</a>
	</div>
	<div>
	<a href='.$linkToUse.'>
	<button style="width: 250px; height: 50px; border: none;
	border-radius: 4px; padding: 18px; font-family: "Rubik", sans-serif;
	font-size: large; cursor: pointer; background: #2d3b55; color: #fff;
	letter-spacing: 0.2px; outline: 0; position: relative; top: 50\%;
	left: 50\%; -ms-transform: translate(-50\%, -50\%); transform: translate(-50\%, -50\%);">
	Verifica la tua Mail
	</button>
	</a>
	</div>
	</form>
	</div>
	</body>	</html>';
}
 \end{lstlisting}
 
 Ultima è stata la definizione del corpo alternativo nel caso in cui non si voglia visualizzare la mail con il corpo "html".\\
 
  \begin{lstlisting}[language=php]
 public function getAlternativeBody($linkToUse){
 	return 'Usa questo link per accedere alla pagina di verifica della mail: '.$linkToUse;
 }
  \end{lstlisting}
  
  Sono stati inoltre scritti dei metodi aggiuntivi per facilitare l'interazione con "l'url" da usare per verificare la mail:\\
  
 \begin{lstlisting}[language=php]
   	public function getUrl($myData){
   		return self::URL."?".http_build_query($myData);
   	}
   	
   	public function getInfoFromUrl($url){
   		$url_component = parse_url($url);
   		parse_str($url_component['query'], $params);
   		return $params;
   	}
 \end{lstlisting}
 
 Per spiegare correttamente il funzionamento della logica viene riportato inferiormente la parte di codice della pagina di registrazione che sfrutta i metodi presentati sopra per mandare la mail di verifica dopo la registrazione. 
 
 \begin{lstlisting}[language=php]
 $mailUtil = new MailUtilities();
 
 $tempArray = array("email" => $email);
 
 $URLToSend = $mailUtil->getURL($tempArray);
 
 $utilities->SendEmail($_POST['email'], $mailUtil->getSubject(), $mailUtil->getBody($URLToSend), $mailUtil->getAlternativeBody($URLToSend));
 /*--- Return to main page ---*/
 $_SESSION['user'] = "Registrated";
 header("Location: main.php");
 exit;
  \end{lstlisting}
  
  In pratica (dopo aver controllato la consistenza dei dati nel database) come si può facilmente leggere, la mail di destinazione viene presa direttamente dal campo di testo della pagina, l'oggetto della mail da inviare è semplicemente prelevato, mentre "l'url" che verrà usato nel corpo della mail (generato usando il metodo "getUrl") sarà l'unione del link base per poter accedere al sito con in più (come argomento) la mail cifrata dell'utente stesso da usare per fare poi l'accesso al database dalla pagina di verifica della mail del sito.\\
  In particolare (come poi mostrato dal codice riportato inferiormente) per favorire questo processo, la prima pagina in assoluto che viene caricata quando si fa l'accesso alla "web app" non è la pagina di accesso, ma bensì una pagina che a seconda se nell'"url" con il quale si fa l'acceso al sito è presente un argomento o meno, fa poi un "redirect" alla pagina apposita.\\
  
 \paragraph{File index.php}\leavevmode\\
  
 \begin{lstlisting}[language=php]
 <?php
 require 'Quiz-Patente-Nautica/Pagina_Web/ValidationMail/mailUtilities.php';
 
 session_start();
 
 $urlUtility = new MailUtilities();
 $array = $urlUtility->getInfoFromUrl($_SERVER['REQUEST_URI']);
 
 if( $array['email']!== null){	
 	
 	$_SESSION['email'] = $array['email'];
 	header("location:Quiz-Patente-Nautica/Pagina_Web/ValidationMail/verificationMail.php");
 	exit;
 }else{
 	header("Location:Quiz-Patente-Nautica/Pagina_Web/main.php");
 	exit;
 }
 ?>
 \end{lstlisting}
 
 Nel caso in cui nell'"url" ci sia un argomento, ecco il comportamento della pagina di verifica della mail.\\
 
\paragraph{Estratto dalla pagina di verifica della mail: "verificationMail.php"}\leavevmode\\
 
 \begin{lstlisting}[language=php]
 /*verify if mail verification has not been did yet*/
 
 $query = $utilities->getMysql()->query("SELECT verified FROM user_table1 WHERE (email = '{$_SESSION['email']}')");
 $tempArray = $query->fetch_array(MYSQLI_ASSOC);
 
 //redirect in positive case
 
 if($tempArray['verified'] == true){
 	$_SESSION['user'] = "Verified";
 	$_SESSION['email'] = null;
 	header("location:../main.php");
 	exit;
 	
 }
 
 $query = $utilities->getMysql()->query("SELECT id FROM user_table1 WHERE (email = '{$_SESSION['email']}')");
 $tempArray = $query->fetch_array(MYSQLI_ASSOC);
 
 /*verify if mail is correct*/
 if($tempArray['id'] == null){
 	$_SESSION['user'] = "NoMail";
 	header("location:../main.php");
 	exit;
 }
 
 /*confirm the mail*/
 
 if(isset($_POST['confirm'])){
 	
 	$temp = $utilities->getMysql()->query("UPDATE user_table1 SET verified = 1 WHERE (id = '{$tempArray['id']}')");
 	
 	if(!$temp){
 		$_SESSION['user'] = "DatabaseError";
 		$_SESSION['email'] = null;
 		header("location:../main.php");
 		exit;
 	}
 	
 	$query = $utilities->getMysql()->query("SELECT verified FROM user_table1 WHERE (email = '{$_SESSION['email']}')");
 	$tempArray = $query->fetch_array(MYSQLI_ASSOC);
 	
 	if($tempArray['verified'] == true){
 		$_SESSION['user'] = "YesMail";
 		$_SESSION['email'] = null;
 		header("location:../main.php");
 		exit;
 		
 	}else{
 		$_SESSION['user'] = "DatabaseError";
 		$_SESSION['email'] = null;
 		header("location:../main.php");
 		exit;
 	}
 }
 \end{lstlisting}
 
 Come si può facilmente notare, prima di tutto viene verificato che l'operazione di verifica non sia già avvenuta, successivamente si aspetta che l'utente premi il pulsante di verifica per poter poi scrivere nel database l'avvenuta verifica.\\
 
 A questo punto della spiegazione è possibile introdurre il codice della pagina di registrazione (pagina raggiungibile solo attraverso quella di accesso tramite la pressione dell'apposito pulsante).\\
 Per prima cosa verrà incluso il codice in "php" in modo da poter capire il suo funzionamento e poi successivamente verrà incluso il codice "html" per mostrare come è strutturata la pagina.\\
 
 \paragraph{registration.php}\leavevmode\\
 
 \begin{lstlisting}[language=php]
 <?php
 //require 'Utilities/keys.php';
 require 'Utilities/utilities.php';
 require 'ValidationMail/mailUtilities.php';
 
 /*get utility available*/
 $utilities = new Utilities();
 
 /*Starting session*/
 session_start();
 
 /*check if  database is reachable*/
 if ($utilities->getMysql()->connect_errno) {
 	die('Impossible conncet to database'.$utilities->getMysql()->connect_error);
 }
 
 if(isset($_POST['register'])){
 	
 	if(filter_has_var(INPUT_POST, 'agree')){
 		
 		if(!empty($_POST['name']) && !empty($_POST['surname']) && !empty($_POST['email']) && !empty($_POST['password1']) && !empty($_POST['password2'])){
 			
 			if($_POST['password1'] == $_POST['password2']){
 				
 				/*extract all fields to use*/
 				$name = $utilities->Encrypt($_POST['name']);
 				$surname = $utilities->Encrypt($_POST['surname']);
 				$email = $utilities->Encrypt($_POST['email']);
 				$password = $utilities->Encrypt($_POST['password1']);
 				
 				/*make query to register new dates*/
 				$query = $utilities->getMysql()->query("INSERT INTO user_table1 (name, surname, email, password) VALUES ('{$name}', '{$surname}', '{$email}', '{$password}')");
 				
 				/*verify if query worked*/
 				if(!$query){
 					die('Query failed'.$utilities->getMysql()->error);
 				}
 				
 				/*test if registration had success */
 				$query = $utilities->getMysql()->query("SELECT password FROM user_table1 WHERE (email = '{$email}')");
 				$TestPassword = $query->fetch_array(MYSQLI_ASSOC)['password'];
 				
 				/*verify if passwords are equal*/
 				if($password === $TestPassword){
 					
 					/*--- Send evrification mail ---*/
 					
 					$mailUtil = new MailUtilities();
 					
 					$tempArray = array("email" => $email);
 					
 					$URLToSend = $mailUtil->getURL($tempArray);
 					
 					$utilities->SendEmail($_POST['email'], $mailUtil->getSubject(), $mailUtil->getBody($URLToSend), $mailUtil->getAlternativeBody($URLToSend));
 					/*--- Return to the main page ---*/
 					$_SESSION['user'] = "Registered";
 					header("Location: main.php");
 					exit;
 					
 				}else{
 					$query = $utilities->getMysql()->query("DELETE * FROM usert_table1 WHERE (email = '{$email}')");
 					
 					if(!$query){
 						die('Query failed'.$utilities->getMysql()->error);
 					} 
					$utilities->Popup("La registrazione ha dato esito negativo");
 				}
 			}else{
 				$utilities->Popup("Le password non corrispondono");
 			}
 		}else{
 			$utilities->Popup("I campi devono essere tutti riempiti");
 		}
 	}else{
 		$utilities->Popup("Devi accettare i termini di utilizzo");
 	}
 }
 
 /*Return to main page page*/
 if(isset($_POST['cancel'])){
 	header("Location: main.php");
 	exit;
 }
 
 ?>
 \end{lstlisting}
 
 Si può subito notare osservando le classi incluse, che è presente quella che contiene i metodi per la gestione della mail, in modo da poter (subito dopo aver completato la prima fase di registrazione) inviare la mail per fare la verifica dell'utente.\\
 Viene fatta iniziare la sessione siccome nello spostamento tra le varie pagine è necessario tenere in memoria lo stato della sessione, in modo da avere sempre chiaro lo stato dell'utente durante la navigazione.\\
 Siccome questa pagina fa uso del database, si controlla che la connessione a quest'ultimo sia possibile.\\
 A questo punto compare una serie di "if" annidati che anche se rendono la lettura del codice più difficile, offrono una maggiore garanzia che la pagina non abbia dei comportamenti anomali, cosa che potrebbe accadere nel momento in cui ogni "if" non fosse annidato ma posto in sequenza.\\
 Si desidera far notare che a fine registrazione viene impostata la sessione utente allo stato "registered" in modo da poter poi far apparire nella pagina principale il messaggio di registrazione avvenuta.\\
 
 \begin{lstlisting}[language=html]
 	<!DOCTYPE html>
 	<html lang="it" >
 	<head>
 	<meta charset="UTF-8">
 	<title>Registrazione</title>
 	<link rel='stylesheet' href='https://fonts.googleapis.com/css?family=Rubik:400,700'><link rel="stylesheet" href="style.css">
 	
 	</head>
 	<body>
 	<!-- Welocome text -->
 	<div class="welcomeText">
 	<h1>Quiz patenti nautiche DD 131 del 31 Maggio 2022</h1>
 	</div>
 	<!-- Login form -->
 	<div class="login-form">
 	<form method="post">
 	<h1>Inserisci i dati richiesti</h1>
 	<div class="content">
 	<div class="input-field">
 	<input type="text" name="name" placeholder="Nome">
 	</div>
 	<div class="input-field">
 	<input type="text" name="surname" placeholder="Cognome">
 	</div>
 	<div class="input-field">
 	<input type="email" name="email" placeholder="Email">
 	</div>
 	<div class="input-field">
 	<input type="password" name="password1" placeholder="Password">
 	</div>
 	<div class="input-field">
 	<input type="password" name="password2" placeholder="Ripeti Password">
 	</div>
 	</div>
 	<div class="checkbox">
 	<input type="checkbox" name="agree" value ="agree"/>
 	<label for="agree"> Accetto i <a href="Terms&Services/terms&Services.html">Termini</a> di utilizzo</label>
 	</div>
 	<div class="action">
 	<input type="submit" name="register" value="Registrati" id="button2"/>
 	<input type="submit" name="cancel" value="Annulla" id="button1"/>
 	</div>
 	</form>
 	</div>
 	
 	</body>
 	</html>
  \end{lstlisting}
  
  Osservando la parte in "html" della pagina si possono notare i vari oggetti che sono stati istanziati per essere poi interagiti tramite la parte in "php". Da questo momento in poi, raramente verrà incluso il codice "html" delle pagine siccome quest'ultimo è più meno sempre lo stesso. Verranno invece evidenziate soltanto la parti peculiari degne di nota.\\
  
  Vista ora la pagina di registrazione, si può procedere con l'analisi della pagina di accesso.\\
  
  \paragraph{main.php}\leavevmode\\
  
  Questa pagina vuole essere usata come pretesto per mettere in evidenzia che durante l'intero sviluppo della applicazione web è stata prestata particolare attenzione alla scrittura di un codice efficiente in modo non solo di poter risparmiare risorse sul "RaspberryPi" ma anche con l'intento di rendere la navigazione più "responsive" e dove è stato possibile, si è cercato di ridurre il minimo le interazioni tra "server" e "client" con l'intento di alleggerire il costo delle varie connessioni in modo poi da poter garantire a più utenti una esperienza soddisfacente. Nel caso specifico di questa pagina (come sottolineato dal codice riportato inferiormente) la gestione dei veri avvisi è stata fatta utilizzando lo "switch" rispetto ad una serie di "if" in cascata, tenendo presente che il comportamento del sito non sarebbe cambiato a prescindere dall'uso di uno dei due sistemi, in questo modo si è visto (utilizzando l'applicazione "\href{https://htop.dev/}{HTOP}") che l'accesso al sito pesava meno sul "core" della macchina che è stato scelto dal sistema operativo per elaborare la richiesta.\\
  
 \begin{lstlisting}[language=php]
 /*reset latest connection*/
 $_SESSION['mail'] = null;
 $_SESSION['password'] = null;
 
 /*check if database connection is okay*/
 if ($utilities->getMysql()->connect_errno) {
 	die('Unable to connect to database'. $utilities->getMysql()->connect_error);
 }
 
 /*Verify status*/
 switch($_SESSION['user']){
 	case "Registered":
 	$utilities->Popup("La registrazione e' avvenuta con successo");
 	$utilities->Popup("La mail per verificare l`account e stata inviata");
 	$_SESSION['user'] = null;
 	break;
 	
 	case "MailSend":
 	$utilities->Popup("La mail per verificare l`account e' stata inviata");
 	$_SESSION['user'] = null;
 	break;
 	
 	case "NoMail":
 	$utilities->Popup("Non e' stato possibile confermare la mail");
 	$_SESSION['user'] = null;
 	break;
 	
 	case "YesMail":
 	$utilities->Popup("La mail e' stata confermata con successo");
 	$_SESSION['user'] = null;
 	break;
 	
 	case "DatabaseError":
 	$utilities->Popup("Errore durante l`aggiornamento del database");
 	$_SESSION['user'] = null;
 	break;
 	
 	case "Verified":
 	$utilities->Popup("La mail e' gia' stata verificata");
 	$_SESSION['user'] = null;
 	break;
 	
 	case "YesDelete":
 	$utilities->Popup("L`account e' stato eliminato con successo");
 	$_SESSION['user'] = null;
 	break;
 	
 	case "NoDelete":
 	$utilities->Popup("Non e' stato possibile eliminare l`account");
 	$_SESSION['user'] = null;
 	break;
 	
 	case "NotAllow":
 	$utilities->Popup("Non si possiedono i permessi per poter accedere alla pagina");
 	$_SESSION['user'] = null;
 	break;
 }
\end{lstlisting}

 Viene ora presentata a titolo informativo (perché non vi si trova qualcosa di peculiare da commentare) la parte di codice che si occupa di gestire l'autenticazione dell'utente.\\ 

\begin{lstlisting}[language=php]
	/*when login button is clicked*/
	if(isset($_POST['enter'])){
		
		if(!empty($_POST['email']) && !empty($_POST['password'])){
			
			$email = $utilities->Encrypt($_POST['email']);
			$query = $utilities->getMysql()->query("SELECT password, verified FROM user_table1 WHERE (email = '{$email}')");
			
			/*check query correctness*/
			if(!$query){
				$utilities->Popup("Errore durante l'accesso alle informazioni");
				exit;
			}
			
			/*extract my aray*/
			$tempArray = $query->fetch_array(MYSQLI_ASSOC);
			
			$verified = $tempArray['verified'];
			/*extract the password from query*/
			$password = $tempArray['password'];
			
			/*verify account esistence*/  
			if($password !== null){
				
				/*check mail verification*/
				if($verified !== null && $verified == true){
					
					/*verify password correctness*/
					if($utilities->Decrypt($password) === $_POST['password']){
						$_SESSION['mail'] = $email;
						$_SESSION['password'] = $password;
						header("Location: Course/entry.php");
						exit;
						
					}else{
						$utilities->Popup("Le credenziali inserite non sono corrette");
					}
					
				}else{
					$utilities->Popup("La mail dell`account non e stata verificata.");
				}
				
			}else{
				$utilities->Popup("L`account non esite");
			}
			
		}else{
			$utilities->Popup("I campi devono essere tutti riempiti");
		}
	}
\end{lstlisting}

Finora è stato mostrata la registrazione e l'autenticazione dell'utente, ma nel sito è anche disponibile una pagina che permette ("on-fly") della quale viene presentato velocemente il codice "php".\\

\paragraph{maintenance.php}\leavevmode\\

\begin{lstlisting}[language=php]
	/*function to verify if credentials are okay*/
	function verifyAccount(){
		
		if(!empty($_POST['email']) && !empty($_POST['password'])){
			
			$email = $GLOBALS['utilities']->Encrypt($_POST['email']);
			$query = $GLOBALS['utilities']->getMysql()->query("SELECT password FROM user_table1 WHERE (email = '{$email}')");
			
			/*verify query correctness*/
			if(!$query){
				$GLOBALS['utilities']->Popup("Errore nel accedere alle informazioni");
				exit;
			}
			
			/*extract password*/
			$password = $query->fetch_array(MYSQLI_ASSOC)['password'];
			
			/*verify account existence*/
			if($password !== null){
				
				if($GLOBALS['utilities']->Decrypt($password) === $_POST['password']){
					
					return true;
					
				}else{
					$GLOBALS['utilities']->Popup("Le credenziali inserite non sono corrette");
					return false;
				}
				
			}else{
				$GLOBALS['utilities']->Popup("L`account non esite");
				return false;
			}
			
		}else{
			$GLOBALS['utilities']->Popup("Per procedere con l`operazione i campi devono essere tutti riempiti");
			return false;
		}
		
	}
	
	/*set status in case of account deleting*/
	if(isset($_POST['delete'])){
		
		if(verifyAccount()){
			
			/*at this point is safe remove account*/
			$email = $utilities->Encrypt($_POST['email']);
			$query = $utilities->getMysql()->query("DELETE FROM user_table1 WHERE (email = '{$email}')");
			
			if(!$query){
				$utilities->Popup("Errore durante l`eliminazione dell`account");
				exit;
			}
			
			$query = $utilities->getMysql()->query("SELECT password FROM user_table1 WHERE (email = '{$email}')");
			$password = $query->fetch_array(MYSQLI_ASSOC)['password'];
			if($password == null){
				$_SESSION['user'] = "YesDelete";
				header("Location: ../main.php");
				exit;
			}else{
				$_SESSION['user'] = "NoDelete";
				header("Location: ../main.php");
				exit;
			}
		}
		
	}
	/*In case to resend the verification email*/
	if(isset($_POST['resend'])){
		
		/*verify account*/
		if(verifyAccount()){
			$email = $utilities->Encrypt($_POST['email']);
			
			$tempArray = array("email" => $email);
			
			$URLToSend = $mailUtil->getURL($tempArray);
			
			$utilities->SendEmail($_POST['email'], $mailUtil->getSubject(), $mailUtil->getBody($URLToSend), $mailUtil->getAlternativeBody($URLToSend));
			
			$_SESSION['user'] = "MailSend";
			header("Location: ../main.php");
			exit;
			
		}
	}
\end{lstlisting}

Una peculiarità di questa pagina è il fatto che come premesso, vengano fatte delle operazioni "on-fly". La motivazioni dietro questa scelta stanno nella gestione semplificata delle operazioni più critiche (siccome è possibile riciclare del codice) oltre che in questo modo si garantisce una maggiore sicurezza, in quanto non si deve gestire una sessione utente "critica" nella quale sarebbe possibile tentare di fare un attacco di "Hijacking". La non presenza di una sessione con tanto di pagina di "atterraggio" nella quale poter poi effettuare le operazioni di gestione dell'account previene questo tipo di attacchi.\\
Come si può leggere da codice, questa pagina mette a disposizioni due principali operazioni: quella di farsi mandare nuovamente la mail per la verifica dell'utente e la possibilità di poter cancellare l'account.\\
A questo punto si vuole evidenziare come la scelta progettuale fatta risulti particolarmente efficace. Supponiamo che un utente volesse fare il rinvio della mail di verifica, questa operazione implica che la mail dell'utente deve rimanere in sessione (dopo aver effettuato l'acceso) in modo che la pagina di manutenzione possa poi eseguire l'operazione. Ma è evidente che è rischioso tenere in sessione dei dati sensibili dell'utente. Per questo motivo le operazioni "on-fly" quindi con ri-autenticazione offrono un margine di sicurezza maggiore.\\