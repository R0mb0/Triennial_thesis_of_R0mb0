\chapter{Programmazione del servizio online parte 2}\label{cap:Programmazione del servizio online parte 2}

\section{Creazione del sito internet}\label{sez:Creazione sito internet}

La creazione del dito internet si può suddividere nella creazione delle singole pagine che lo compongo.

\subsection{Creazione pagina di registrazione e di accesso}
L'intera produzione del sito internet è stata fatta seguendo la filosofia di dare più importanza ai lavori "più grossi", facendosi che quindi il progetto si sia sviluppo a partire da un nucleo principale per poi espandersi, al contrario di  come comunemente vengono condotti questo genere di progetti, partendo dalle specifiche più esterne (in modo da avere più velocemente una idea del prodotto finito) per poi convergere verso il nucleo.\\
Secondo la specifica del progetto, il sito deve essere in grado di essere accessibile solo agli utenti registrati (verificando inoltre che la mail che abbiano inserito sia reale), garantendo inoltre la sicurezza, ossia che il sito deve possedere le principali difese per non essere attaccato e che in caso di attacco i dati degli utenti siano all'interno del "database" cifrati in modo da non poter essere utilizzati da terzi.\\
La prima pagina che è stata creata in assoluto è quella di accesso, dalla quale poi premendo un  