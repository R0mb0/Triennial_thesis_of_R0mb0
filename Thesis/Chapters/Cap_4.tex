\chapter{Programmazione del servizio online parte 2}\label{cap:Programmazione del servizio online parte 2}

\section{Creazione del sito internet}\label{sez:Creazione sito internet}

La creazione del dito internet si può suddividere nella creazione delle singole pagine che lo compongo.

\subsection{Creazione pagina di registrazione e di accesso}
L'intera produzione del sito internet è stata fatta seguendo la filosofia di dare più importanza ai lavori "più grossi", facendosi che quindi il progetto si sia sviluppo a partire da un nucleo principale per poi espandersi, al contrario di  come comunemente vengono condotti questo genere di progetti, partendo dalle specifiche più esterne (in modo da avere più velocemente una idea del prodotto finito) per poi convergere verso il nucleo.\\
Secondo la specifica del progetto, il sito deve essere in grado di essere accessibile solo agli utenti registrati (verificando inoltre che la mail che abbiano inserito sia reale), garantendo inoltre la sicurezza, ossia che il sito deve possedere le principali difese per non essere attaccato e che in caso di attacco i dati degli utenti siano all'interno del "database" cifrati in modo da non poter essere utilizzati da terzi.\\
La prima pagina che è stata creata in assoluto è quella di accesso, dalla quale poi premendo un  pulsante è possibile accedere alla pagina di registrazione. Com'è deducibile queste due pagine sono state in contemporanea siccome la prima permette di verificare li corretto funzionamento della seconda.\\

Dopo aver studiato velocemente "l'html" e il "php" (il "javascript" è stato lasciato in secondo piano) decido di sviluppare il codice ("php") con paradigma ad oggetti, siccome mi è molto famigliare. Ragionando solo a livello di logica decido di creare una classe "utilities" nella quale poter scrivere i metodi che poi mi torneranno utili durante anche lo sviluppo delle restanti parti. \\

\paragraph{Classe utilities}
La classe "utilities" (quindi il relativo file) è stata implementata in questo modo:\\
Per prima si è deciso di dividere in file file separati i metodi e le chiavi per il loro funzionamento. Questa cosa è stata fatta prevalentemente per ragioni di ordine del codice e per aumentare la sicurezza.\\
Il primo metodo presente è quello che permette di creare l'oggetto "mysqli" che serve per poter interagire con il database (da notare come nella porzione di codice inferiore non vi siano riportati direttamente i dati di accesso). Essendo evidente che questo oggetto è per sua stessa statico, durante l'implementazione del metodo si è usato il "singleton pattern" per evitare che in memoria vi siano multipli oggetti identici.\\

\begin{minted}{php}
	/*using singleton pattern*/
	public function getMysql(){
		if($this->mysqli == null){
			$this->mysqli = new mysqli(SERVER, USER, PASSWORD, DATABASE);
			
		}
		return $this->mysqli;
	}
\end{minted}