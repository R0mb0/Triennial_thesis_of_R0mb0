\chapter{Premessa}\label{cap:premessa}

\section{Com'è stata scritta la tesi}\label{sez:Scrittura tesi}

La scrittura di questa premessa riguarda la presentazione di un lavoro che è stato svolto durante la stesura della tesi in \LaTeX, ovvero quello di scrivere tutta una serie di "override" per aumentare la capacità espressiva e soprattutto per poter includere nel documento un codice che non fosse solo indentato ma che avesse anche la colorazione della sintassi in modo da facilitarne la lettura.\\
Uno dei primi problemi riscontrati durante la stesura della tesi è che il "template" messo a disposizione dall'università non prevedeva un "enviroment" per poter includere un "listato di codice" di bell'aspetto all'interno del documento.\\
Quindi per prima cosa si è provato ad usare "l'enviroment" di base del \LaTeX, ovvero "verbatim", ma quest'ultimo non faceva altro  che cambiare il font del testo (impostando come font quello "macchina") e dare un maggior peso alla indentazione rispetto a quello dato da "pdftex". Il codice così presente nel documento era un listato indentato in bianco e nero. Questo tipo di presentazione può andare bene per piccoli listati di codice molto semplici, dato che la mancanza della colorazione della sintassi rende la lettura più difficile, ma per linguaggi più complessi come "l'html", il "css", "php" e "javascript" (i linguaggi che sono stati riportati nel documento) la colorazione della sintassi è una cosa fondamentale per potersi orientare lungo il codice.\\
Esiste per il \LaTeX un "enviroment" chiamato "lstlisting" che consente non solo di presentare un listato indentato, ma anche di avere la colorazione della sintassi in base al tipo di linguaggio scelto e in base ai colori indicati, per esempio scegliendo come linguaggio il "Python" è possibile impostare che i commenti siano di colore blu e le stringhe di colore verde.\\
Il problema principale con l'omonima libreria è che quest'ultima non viene più aggiornata da molto tempo e che è stata rilasciata incompleta. Purtroppo tra i linguaggi non mappati c'erano proprio quelli che si volevano riportare nel documento. Quindi si è dovuto mappare manualmente i vari linguaggi facendo un "override" delle impostazioni di base "dell'enviroment". Questa procedura è stata fatta seguendo quel poco di documentazione ufficiale che si è riusciti a reperire e leggendo i forum su internet per vedere le varie soluzioni proposte.\\
Secondo la documentazione ufficiale la libreria permetteva (a livello di "header" del documento \LaTeX) di poter mappare più linguaggi di programmazione simultaneamente e non uno solo come spesso veniva riportato nei forum, ma durante le prove di scrittura del documento si è visto che effettivamente era vero che la libreria era stata pensata per svolgere questo tipo di operazioni, ma in realtà ciò non aveva mai funzionato (si fa riferimento al motore di compilazione "pdftex") ed è per questo motivo che si è adottata la soluzione di fare un "override" a cascata. L'idea era quella (sfruttando dei bug presenti nella libreria) di mappare accodato ogni linguaggio come se dovesse essere l'unico presente nel documento. Il risultato ottenuto è che effettivamente si poteva disporre della mappatura di più linguaggi simultaneamente, ma la colorazione della sintassi ereditata non era quella indicata al momento della mappatura, bensì l'ultima indicata in assoluto rispetto all'ordine dei linguaggi. Un'altra incongruenza riscontrata è che in questo modo la libreria faceva una specie di "merge" delle impostazioni dei vari linguaggi, per esempio "nell'html" non è importante colorare i numeri in modo diverso dal testo, cosa che viene fatta ad esempio nel "php", ma con il fatto "dell'override" a cascata anche il listato del "nell'html" viene presentato con i numeri colorati in modo diverso rispetto al testo.\\
Senza riportare in questa sede il codice scritto "nell'header" del documento \LaTeX si segnala che il codice è riportato su "GitHub" al seguente link: \url{https://github.com/R0mb0/Triennial_thesis_of_R0mb0}

Un altro lavoro fatto è stato "l'override" del "template" in \LaTeX fornito dall'università in modo da poter migliorare l'estetica del documento. In particolare si sono dovuti gestire manualmente i problemi tra le parti "float" e le parti "static" presenti. Il problema riguardava proprio come il "template" fornito andava ad impaginare, ovvero veniva data precedenza agli ambienti "float" rispetto agli altri. Questa è una ottima scelta nel caso in cui il "template" venga utilizzato da chi non ha dimestichezza con il linguaggio \LaTeX ma nei restanti casi è un forte limite, sopratutto perché in questa situazione è facile che i vari elementi del documento si vadano a sovrapporre  (tipo del testo sopra delle immagini che non sono "float"). In questo caso la soluzione è stata quella di rompere la geometria della pagina usando la libreria "geometry" e successivamente sono state "protette" tutte le varie parti del documento usando le "minipage". Quindi con la geometria "rotta" le pagine mettevano a disposizione un volume maggiore da poter utilizzare (questo previene la sovrapposizione degli elementi) e con le parti del documento protette si poteva fare manualmente l'impaginazione utilizzando i modificatori di volume "hspace*" e "vspace*". 