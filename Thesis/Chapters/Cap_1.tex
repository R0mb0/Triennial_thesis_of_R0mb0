\chapter{Introduzione}\label{cap:Introduzione}

\section{Analisi preventiva del problema}\label{sez:Analisi preventiva}

L'obiettivo del progetto è quello di creare una demo di un sito internet (da pubblicare una volta completato) che permetta di potersi esercitare sui quiz "DD 131 del 31 maggio 2022" della patente nautica entro e oltre le 12 miglia, per imbarcazioni sia a vela che a motore.\\
L'analisi preventiva del problema, per tanto, si è composta delle seguenti azioni: Il reperimento dei documenti riguardanti il nuovo decreto; il reperimento dei quiz conformi al "DD 131 del 31 maggio 2022"; la selezione di un "single-board" nel quale fare le prove; la comprensione di come poter programmare efficacemente il dispositivo scelto in relazione al tipo di operazioni che deve svolgere; la pianificazione della progettazione di un database efficiente per la gestione delle domande della prova.\\

\subsection{Reperimento dei documenti riguardanti il nuovo decreto}
Il reperimento dei documenti riguardante il nuovo decreto è stato fatto avvalendosi del sito della Capitaneria di Pesaro e del sito del Ministero delle Infrastrutture e dei Trasporti. Dal primo si pensava di scaricare i documenti sulle regole d'esame (come il numero delle prove, il tempo limite per ogni prova e il tipo e il numero di domande); dal secondo, invece,  di scaricare l'intero documento relativo ai quiz d'esame, ma in entrambi i siti non vi erano i documenti cercati . Ho deciso di visionare il sito della Capitaneria di Pesaro perché in passato ogni "compartimento marino" aveva la facoltà di modificare leggermente le regole d'esame, come per esempio le domande presentate al candidato e il tempo della prova; ora invece  il nuovo regolamento stabilisce che le domande devono essere le stesse in tutto il territorio nazionale senza possibilità di modifica (ma non è stato specificato se ancora i "dipartimenti marini" abbiano la facoltà di personalizzare alcune parti dell'esame).\\
Quindi una delle prime difficoltà riscontrata, è stata la totale assenza di documentazione da parte degli enti preposti alla diffusione, infatti  il sito del Ministero delle Infrastrutture e dei Trasporti non presentava al proprio interno alcun collegamento per poter scaricare il documento "pdf" contenente le domande del nuovo esame (al tempo della scrittura di questo documento, il problema descritto è stato risolto). Navigando sul sito nella sezione "Nautica" venivano solo presentati alcuni articoli (molto brevi e non tecnici)  sull'applicazione del nuovo regolamento per l'esame. Invece ho trovato il "pdf" contente le domande che cercavo indicizzato su "Google". L'indirizzo  tramite il quale è stato possibile scaricare il documento faceva parte sempre del sito del Ministero delle Infrastrutture e dei Trasporti, ma nell'indirizzo erano presenti collegamenti ai quali non era possibile accedere internamente al sito. Modificando manualmente l'indirizzo  si poteva al massimo riaccedere alla pagina relativa agli articoli sull'esame, siccome tutte le altre pagine non venivano caricate.\\
nche nel sito della Capitaneria di Pesaro non erano presenti le informazioni per l'esame in quanto nella sezione "Patenti" vi era solo scritto che l'esame per il conseguimento della patente nautica era cambiato con un nuovo regolamento, ma non vi era traccia delle informazioni tecniche riguardanti il nuovo esame.\\
È stato quindi ampliato lo spettro d'indagine, ricercando queste informazioni nei siti delle altre capitanerie (in particolare quelle di Cesenatico e di Genova) e nei siti delle scuole nautiche nazionali.\\
Nel sito della Capitaneria di Cesenatico ho trovato l'informazione che il nuovo esame era unico per tutta Italia e questo mi ha permesso di poter accettare anche le informazioni trovate al di fuori del mio Compartimento.\\
Confrontando i vari documenti trovati nei siti delle Capitanerie mi sono accorto che molti di essi erano in contraddizione su alcuni temi come ad esempio il numero di domande per prova. Le informazioni sono state riordinate confrontandomi con persone esperte di nautica e in base a ciò che era più verosimile rispetto alla storia dell'esame. Ho prodotto un documento (reperibile nel "Repository" del progetto) che dovrebbe descrivere con un alto grado di correttezza l'attuale regolamento per il nuovo esame della patente nautica.\\